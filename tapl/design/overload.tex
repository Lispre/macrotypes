\documentclass{article}
\usepackage{mathpartir}
\usepackage{palatino}

\newcommand{\fmt}[1]{\mathsf{#1}}
\newcommand{\ksig}{\fmt{signature}}
\newcommand{\kinst}{\fmt{instance}}
\newcommand{\kres}{\fmt{resolve}}
\newcommand{\kpsi}[3]{\psi~(#1)~#2~#3}

\begin{document}

\section*{Parametric Overloading}

\subsection*{Definition}
\begin{itemize}
\item Identifiers may be parametrically overloaded.
\item Overloaded identifiers have no implementation and a type containing holes.
\item Implementations may be \emph{later} associated with the identifier by filling the type holes.
\item On use, the identifier infers from its context which instance to use.
  Instances are keyed by type: the type filling the hole.
\end{itemize}

Goal: provide a clean implementation.

\subsection*{Old Ideas}
\subsubsection*{Global Table}
\begin{itemize}
\item Keep a global mapping $\Sigma$ from identifiers to types to implementations.
\item Query and extend this map during typechecking.
\end{itemize}

Problem: global state like this is anti-modular.
Unclear how to extend to a multi-module or sub-module world.
There's only a global scope.

\subsubsection*{Dynamic Binding \& Parameters}
\begin{itemize}
\item Each overloaded identifier represented as a new \emph{parameter}
\item The parameter is a lookup procedure; using the ID should trigger a type-directed lookup.
  Somehow.
\item New instances extend the lookup procedure in the parameter.
\end{itemize}

Parameters are a nice mix of lexical scope and imperatives.
Also they are reasonable to share across modules.

The lookup is a little troublesome, but the compile-time elaboration should be able to call the parameter with the right type arguments.


\newpage
\subsection*{Current Idea: first class overloadables}
We'll represent overloadable signatures with a new type constructor, $\psi$, which makes the updating explicit.

The $\psi$ types have the form
$$\kpsi{\alpha}{\Sigma}{\tau}$$
where:
\begin{itemize}
\item $\alpha$ is a type variable
\item $\Sigma$ is a partial map from types to expressions.
  It is a lookup table: given a concrete type $\tau'$ to swap for the variable $\alpha$, the map $\Sigma$ returns an expression with type $\tau[\alpha/\tau']$.
\item $\tau$ is a type where $\alpha$ is definitely free
\end{itemize}

For now, we constrain $\tau$ to be a function $\alpha \rightarrow \tau'$ for some base type $\tau'$.

In the following rules, we identify $\Sigma$ with its set of keys.
These keys are the carrier set for the algebra of types at which we can instantiate the $\psi$ type.
%% TODO explain the carrier a little better

At runtime, all values of $\psi$ type are dictionaries that perform type dispatch.
Applying a $\psi$-value triggers a series of predicate checks on its argument; if successful, this leads to a function application.

\begin{mathpar}
  \inferrule*[left=O-Sig]{
    \alpha\ \mbox{free in}\ \tau
    \\\\
    \mbox{(for now, $\tau = \alpha \rightarrow \tau'$)}
    \\\\
    \Sigma = \emptyset
  }{
    \vdash \ksig\ \alpha\ \tau : \kpsi{\alpha}{\Sigma}{\tau} \Rightarrow \Sigma
  }

  \inferrule*[left=O-Inst]{
    \vdash x : \kpsi{\alpha}{\Sigma}{\tau'}
    \\\\
    \tau \not\in \Sigma
    \\\\
    \vdash e : \tau'[\alpha/\tau]
  }{
    \vdash \kinst\ x\ \tau\ e : \kpsi{\alpha}{(\Sigma \cup \{\tau\})}{\tau'}
    \Rightarrow (\mathsf{set!}\ \Sigma\ ???)
  }

  \inferrule*[left=O-App]{
    \vdash e : \tau
    \\
    \vdash (\kres\ x\ \tau)~e : \tau'' \Rightarrow v
  }{
    \vdash x~e : \tau''
    \Rightarrow v
  }

  \inferrule*[left=O-Res]{
    \vdash x : \kpsi{\alpha}{\Sigma}{(\alpha \rightarrow \tau'')}
    \\
    (\tau, e) \in \Sigma
    \\
    \vdash e : \tau \rightarrow \tau'' \Rightarrow v
  }{
    \vdash \kres\ x\ \tau : \tau \rightarrow \tau'
    \Rightarrow v
  }

\end{mathpar}
We could also do $\kres$ as a metafunction, but this was language users can specify exactly how to resolve their overloaded functions.

\subsection*{Problem!}
Oh wow, I didn't think about $\lambda$ being a problem.
We have types-depending-on-values, but I didn't see any issue because we know everything about these values statically.
Not quite!
$$(\lambda\,(x : \kpsi{\alpha}{\Sigma}{\tau})\,.\,(x~1))~(\kinst~(\ksig~(\alpha)~\tau)~\tau'~e)$$
What's the type of $x$ inside the body?
We know from ``$\Sigma$'' what keys it has, but the overloading-resolver is OUTSIDE the $\lambda$.

Ideas to fix:
\begin{itemize}
\item Give up. Only do run-time instance resolution. HAH.
\item Give up. Fall back to parameter-style or global-table-style. HM.
\item Do some best-effort static analysis to push the type outside the lambda inside it.
  This could be combined with type inference on the $\lambda$ signature\textemdash we really don't care the exact type of $x$ (i.e. its entire carrier), only that it can be applied to integers.
\end{itemize}



\subsection*{Next Steps}
\begin{itemize}
\item Implement with simple types
\item Implement with recursive types like nested products and lists
\item Combine with occurrence typing
\item Extend to multiple variables, $(\alpha^* \ldots)$
\item Extend to overloaded codomains
\item Extend to other overloadables, not just functions
\item Allow bound polymorphism over signatures with at least some instances
\item Work out interactions with $<:$ and $\forall$
\end{itemize}


\end{document}
